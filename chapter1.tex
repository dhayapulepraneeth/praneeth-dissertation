
\chapter{Introduction}
The internet is experiencing significant growth in this fast-paced era. People all around the world are using this medium to connect with each other, which makes communication really easy and smooth. The internet has a global reach, connecting people from all around the world and making distances seem shorter. In today's modern society, technology plays a significant role in our everyday routines. Almost every aspect of our lives, including the things we use and the way we go about our daily activities, is closely connected to technological progress.

One interesting observation in the current era of technology is the significant increase in various online activities, such as banking and shopping. This trend has been pointed out by \cite{Aljabri2022DetectingMU}. On the other hand, the rise in internet usage has also raised worries regarding cybersecurity breaches. Cybercriminals take advantage of weaknesses in the systems of internet users in order to obtain access to important information. One main way that cyber attacks happen is by using malicious Uniform Resource Locator's (URLs). These URLs can trick users into dangerous situations without them realising it.

The task of ensuring user privacy and data integrity has become quite difficult for cybersecurity teams because of the large number of users. The increase in cyber attacks can mostly be linked to the widespread presence of these harmful URLs. URLs play a crucial role in this scenario as they act as virtual addresses that bring websites to our screens. When users enter a URL, it directs them to the desired website and provides them with the relevant information. According to\cite{Anjali}, a typical URL structure consists of a protocol and hostname, which are presented as <protocol><hostname><path>. The path component is used to indicate the specific webpage or resource.

Please take a look at the following example: https://outlook.office.com/mail/ In this case, the "https:" is used to indicate the protocol being used, while "outlook.office.com" serves as the hostname. It is crucial to acknowledge that websites vary in terms of their security measures and user-friendly interfaces. Malicious websites have the ability to trick users into unknowingly sharing private and sensitive information with individuals who intend to cause harm. Cybercriminals are always coming up with new ways to create spam URLs, which they use to trick people who aren't aware of their schemes. According to \cite{Aljabri2022DetectingMU}, the changing environment of the internet poses a danger to individuals who use it. This danger could lead to financial harm and harm to the reputation of organisations.


\section{Breif Background Information}
A Uniform Resource Locator (URL) functions as a digital identifier for a specific web resource and is composed of two essential components. The initial component is the protocol identifier, which designates the specific protocol to be utilised, whereas the subsequent component is the resource name. The term "resource name" denotes the specific geographical location associated with the domain name and the IP address of the user. The components are segregated by a colon and a pair of forward slashes.\cite{MOHANTY20231668} as showed in the Figure 1.1



Cybercriminals leverage coding vulnerabilities in order to gain unauthorised access to servers or databases. The incidence of cyber vandalism threats is experiencing a significant upward trend. As a result, scholars are increasingly focusing their efforts on the classification of URLs, recognising its importance in the protection of privacy and security. An exemplar of malevolent conduct can be observed in the context of phishing emails. These attacks entail the manipulation of users into accessing harmful web pages, resulting in the inadvertent disclosure of their personal information. The identification of phishing web links and the proactive mitigation of zero-day malware and phishing attacks are made possible through the utilisation of CANTINA, a content-based methodology.\cite{Anjali}

Web spam refers to a range of strategies employed by specific websites with the intention of deceiving users through the use of fraudulent emails.\cite{Aljabri2022DetectingMU} Upon clicking, these emails direct users to pages that are potentially harmful. This scenario presents potential avenues for malicious actors to exploit individuals by coercing them into revealing sensitive information, thereby facilitating the extraction of financial resources or other confidential data. The methodologies employed in the execution of cyber attacks are multifaceted and encompass various strategies, such as phishing, malware distribution, SQL injections, and other analogous techniques. The task of identifying security breaches remains a difficult endeavour, as attackers constantly adapt their strategies to mislead users.

There are two distinct types of malicious URLs, namely counterfeit URLs and assault URLs. Criminals employ Domain Generation Algorithms (DGA) in order to generate a plethora of counterfeit Uniform Resource Locators (URLs). The substantial magnitude of this quantity poses a significant challenge for law enforcement in effectively mitigating their impact. On the other hand, attack Uniform Resource Locators (URLs) exploit SQL injections\cite{vijayalakshmi2020web} as a means to specifically target servers and disproportionately obtain data from them.\cite{Zhuofan}
\section{Problem Statement}


The emergence of malicious URLs in our modern era is a significant threat to our society. In the field of cybersecurity, it has become quite difficult to identify and prevent harmful URLs. It is important to understand that taking necessary measures can help prevent possible risks and avoid negative outcomes such as financial losses and security breaches. The malevolent URLs are spread to users through different channels like emails, pop-ups, text messages, and advertisements on websites. Users unknowingly download these threats with just one click, causing things to quickly go downhill.

To tackle this problem, our project focuses on using machine algorithms to predict URL classifications. In order to complete this task, we have used a dataset that is publicly available on Kaggle. The dataset includes various categories of URLs, such as benign, defacement, phishing, and malware URLs. The research methodology involved evaluating feature selection models, selecting supervised machine learning algorithms, training and validating these algorithms, adjusting parameters, comparing accuracies, and analysing performance using different metrics.
.

\section{Aims and Objectives}
The main objectives of this project involve investigating an efficient algorithm that produces optimal accuracy and developing a model specifically designed for detecting malicious URLs. In order to achieve these objectives, a series of interrelated tasks will be undertaken.
\begin{enumerate}
	\item Analysing the literature in-depth to develop a thorough understanding of the subject.
	\item The task involves the identification of algorithms that exhibit efficacy in tackling the given challenge. 
	\item Conducting comprehensive testing on the recently developed algorithm utilising authentic real-world data..
	\item Fine-tuning algorithms involves adjusting a range of parameters in order to optimise their performance.
	\item Develop an effective model that accurately detects malicious URLs, it is imperative to prioritise achieving optimal accuracy.. 
\end{enumerate}


\section{Outline of Thesis}




Chapter 1 explores the establishment of the project's fundamental elements, including the identification of the project's URL, the contextual background, and the underlying motivation that led to its creation. The project's goals and objectives are effectively communicated, accompanied by a thorough delineation of the research strategy and a graphical depiction in the form of a Gantt chart.

Chapter 2 extensively scrutinises a comprehensive range of perspectives presented by various authors regarding malicious attacks. A thorough examination is undertaken to investigate different methodologies developed for the detection of these attacks, alongside a meticulous analysis of supervised machine learning algorithms. The scope of the investigation is expanded to include a thorough examination of previous efforts that have employed different methodologies in order to detect and identify malicious URLs.

Chapter 3 provides a comprehensive examination of various methodologies encompassed within the domain of Exploratory Data Analysis. This encompasses various processes including oversampling, data cleaning, tokenization, elimination of stopwords, stemming, and count vectorization. Furthermore, this study conducts a thorough analysis of the evaluation metrics utilised in different algorithms, as well as an investigation into the procedures for refining datasets.

In the following Chapter 4, an extensive examination of copyright considerations related to the dataset is elaborated upon, thus clarifying any uncertainties. This paper provides a comprehensive examination of potential risks and their subsequent impacts, while also presenting a contingency plan designed to effectively mitigate any unforeseen circumstances.

\section{Summary}


Considerable academic research has been conducted to explore the historical development of Uniform Resource Locators (URLs), thoroughly understanding their structural complexities, utilisation patterns, and the crucial matter of security. Through the course of this investigation, a distinct problem statement has arisen, focusing on the potential damage to an organization's reputation and the illicit monetary benefits obtained through the manipulation and deception of users.